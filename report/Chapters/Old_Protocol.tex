%************************************************
\chapter{Protocol}\label{ch:protocol}
%************************************************



\section{Using asymmetric keys to authenticate}

\paragraph{Paring Authenticator and Sibling.} Paring between a users Authenticator $S$ and it's Sibling $As$ is required before the Authenticator can be registered with a service and used to authenticate. First the parties generates and exchanges public/private keys to ensure secrecy and integrity of messages between the parties. All messages send between $A$ and $As$ in the remainder of this section are \textit{signed-then-encrypted} with the given key, and will for brevity not be notarised.

\paragraph{Registering with a Service Provider.}
First the Authenticator $A$ and it's sibling $As$ initiates the registration with the Service Provider $S$ by generating two new public/private key-pairs $\left(k_a,g^{k_a}\right)$ and $\left(k_{as},g^{k_{as}}\right)$ that will used only with the given $S$. The Authenticator now initiate the registration with a key exchange.

All further communication between $A$ and $S$ will be \textit{signed-then-encryp\-ted} using the key-pairs, and will for brevity not be notarised. %The message flow is also depicted in figure~\ref{msc:registration}.
\begin{align}
A   &\rightarrow S      &&  \left\{g^{k_a},g^{k_{as}}\right\}
\end{align}

S now responds with an initial seed $c$ that should be used for later authentication
\begin{align}
S   &\rightarrow A      &&  \left\{g^{k_s}, m = c_s \cdot  \left( g^{k_{as}} \cdot g^{k_a} \right)^{k_s}\right\}\\
A   &\rightarrow As     &&  \left\{g^{k_s}, m' = m \div {\left(g^{k_s}\right)^{k_a}}\right\}
\end{align}

The sibling $As$ thereafter decrypts $c_s = m' \div {\left(g^{k_s}\right)^{k_{as}}}$

\begin{figure}
%\begin{wide}
\begin{msc}{Registration}

\setlength{\instdist}{2.5cm}
\setlength{\actionwidth}{2.6cm}

\declinst{as}{}{$As$} 
\declinst{a}{}{$A$} 
\declinst{s}{}{$S$} 
\mess{$\left\{g^{k_a},g^{k_{as}}\right\}$}{a}{s}
\nextlevel
\action{Generate $c_s$}{s}
\nextlevel[3]
\mess{$\left\{m,g^{k_s}\right\}$}{s}{a}
\nextlevel
\action{Compute $m'$}{a}
\nextlevel[3]
\mess{$\left\{m',g^{k_s}\right\}$}{a}{as}
\end{msc}
%\end{wide}
\caption{}
\label{msc:registration}
\end{figure}

\paragraph{Authentication.}
After the registration the Authenticator $A$ and it's sibling $As$ can authenticate with the service $S$ by sending a token using their secrets $c_s$ and $k_a$. The message flow is depicted in figure~\ref{msc:authentication}.
\begin{align}
A   &\rightarrow S      &&  \left\{ g^{k_a}, t = \left( g^{c_s} \cdot g^{k_s} \right)^{k_a} \right\}
\end{align}

The session with $S$ will be kept alive continuously and after every successful authentication round $As$ and $S$ computes a $c' = hash(c)$ that should be used for the next round of authentication. As the secrets to compute a valid token are distributed between $A$ and $As$, $S$ is convinced that the Authenticator and it's sibling are collaborating.

The service provider $S$ can verify the authenticity of the token received from the Authenticator $A$ by computing
\begin{align*}
check: && t \stackrel{?}{=} \left(g^{k_a}\right)^{c_s} \cdot \left(g^{k_a}\right)^{k_s}
\end{align*}

Because only the holders of either $\left(k_a,g^{c_s},g^{k_s}\right)$ or $\left(k_s,c_s,g^{k_a}\right)$ can compute a valid token, the authenticity of the token is trusted. Note that the holder of $\left(c_s,g^{k_a},g^{k_s}\right)$ cannot compute the token as
it requires the use of discrete logarithmic in the group $g$ which is computationally hard.



%\begin{align*}
%&& \left( g^{c_s} \cdot g^{k_s} \right)^{k_a} \neq \left( g^{k_a} \cdot g^{k_s} \right)^{c_s} \neq \left( g^{c_s} \cdot g^{k_a} \right)^{k_s}
%\end{align*}




\begin{figure}
%\begin{wide}
\begin{msc}{Authentication}

\setlength{\instdist}{2.5cm}
\setlength{\actionwidth}{2.6cm}

\declinst{as}{$\left\{c_s,k_{as}\right\}$}{$As$} 
\declinst{a}{$\left\{g^{k_{as}},g^{k_{s}},k_a\right\}$}{$A$} 
\declinst{s}{$\left\{c_s,k_{s},g^{k_a}\right\}$}{$S$} 
\inlinestart[1.5cm][1.5cm]{exp1}{loop}{as}{s}
\nextlevel
\inlinestart{exp2}{opt}{a}{s}
\nextlevel[2]
\mess{$\left\{alive?, g^{k_s}\right\}$}{s}{a}
\nextlevel
\inlineend*{exp2}
\nextlevel[2]
\mess{$\left\{auth, g^{k_s}\right\}$}{a}{as}
\nextlevel
\action{compute $g^{c_s}$}{as}
\nextlevel[3]
\mess{$\left\{g^{k_s}, g^{c_s}\right\}$}{as}{a}
\nextlevel
\action{compute $h$}{a}
\nextlevel[3]
\mess{$\left\{g^{k_a},h\right\}$}{a}{s}
\nextlevel
\action{$c_{s}' = h(c_s)$}{as}
\action{$c_{s}' = h(c_s)$}{s}
\nextlevel[2]
\inlineend{exp1}
\end{msc}
%\end{wide}
\caption{}
\label{msc:authentication}
\end{figure}


\section{Attempt 2}

S now responds with a tuple of an initial seed $c$ and a shared key $k_{s,as}$ that should be used for later authentication
\begin{align}
S   &\rightarrow A      &&  \left\{g^{k_s}, m = (c_s,k_{s,as}) \cdot  \left( g^{k_{as}} \cdot g^{k_a} \right)^{k_s}\right\}\\
A   &\rightarrow As     &&  \left\{g^{k_s}, m' = m \div {\left(g^{k_s}\right)^{k_a}}\right\}
\end{align}

The sibling $As$ now decrypts $(c_s,k_{s,as}) = m' \div {\left(g^{k_s}\right)^{k_{as}}}$

After the registration the Authenticator $A$ and it's sibling $As$ can authenticate with the service $S$ by sending a token using their secrets $c_s$, $k_{s,as}$ and $k_a$. The message flow is depicted in figure~\ref{msc:authentication}.
\begin{align}
As  &\rightarrow A      &&  \left\{ c' = hash^{k_{s,as}}(c) \right\}\\
A   &\rightarrow S      &&  \left\{ g^{k_a}, t = c' \cdot \left(g^{k_s} \right)^{k_a} \right\}
\end{align}

The service provider $S$ can verify the authenticity of the token received from the Authenticator $A$ by computing
\begin{align*}
check: && t \stackrel{?}{=} hash^{k_{a,s}}(c) \cdot \left(g^{k_a} \right)^{k_s}
\end{align*}

\newpage
\section{A third attempt}


\paragraph{Standard RSA.}

A generator $GenRSA$ is ran to obtain the pair $(N,e,d)$, where 

\begin{itemize}
    \item $N$ is the product of large safe primes $p$ and $q$.
    \
    \item $e$ is chosen at random from $\mathbb{Z}_{\phi(N)}^*$.
    \item $d$ is the inverse of $e$ given as $d = [e^{-1} \text{ mod } \phi(N)]$.
\end{itemize}

Let's define the functions $enc$ and $dec$ as
\begin{align*}
enc(m,e): && m^e \mod N\\
dec(c,d): && c^d \mod N
\end{align*}

Such that $dec\left(enc(m,e),d\right) = m$ if and only if $d$ is the inverse of $e$.

\paragraph{Mediated RSA.} Mediated RSA is similar to normal RSA with the difference that the private key $d$ is split into $n > 0$ segments.
\begin{align*}
&& d = \sum\limits_{i=1}^n d_i \mod \phi(N)
\end{align*}

The shares can be used for distributed decryption as the product of decryption the cipher with all shares is equal to decryption with the original decryption key.
\begin{align*}
&& \prod\limits_{i=1}^n& \quad c^{d_i} \equiv c^d \mod N
\end{align*}

Thus we have a schema where we can encrypt as with normal RSA, but can only decrypt if all key holders collaborate.

\begin{figure}
%\begin{wide}
\begin{msc}{Authentication}

\setlength{\instdist}{2.5cm}
\setlength{\actionwidth}{3cm}

\declinst{as}{$\left\{d_{As}\right\}$}{$As$} 
\declinst{a}{$\left\{d_{A}\right\}$}{$A$} 
\declinst{s}{$\left\{e\right\}$}{$S$} 
\inlinestart[1.75cm][1.75cm]{exp1}{loop}{as}{s}
\nextlevel[2]
\mess{$m_1 = \left\{n\right\}$}{s}{a}
\nextlevel
\mess{$m_2 = \left\{n\right\}$}{a}{as}
\nextlevel
\action{$h = enc(n,d_{As})$}{as}
\nextlevel[4]
\mess{$m_3 = \left\{h\right\}$}{as}{a}
\nextlevel
\action{$h' = enc(n,d_{A})$}{a}
\nextlevel[4]
\action{$n' = h \times h' \mod n$}{a}
\nextlevel[4]
\mess{$m_4 = \left\{ n' \right\}$}{a}{s}
\nextlevel
\action{$dec(n',e) \stackrel{?}{=} n$}{s}
\nextlevel[4]
\inlineend{exp1}
\end{msc}
%\end{wide}
\caption{}
\label{msc:adwda}
\end{figure}


%*****************************************
%*****************************************
%*****************************************
%*****************************************